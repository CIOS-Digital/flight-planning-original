\documentclass[12pt, letterpaper]{article}
% Reduce the page margins to 3/4 of an inch
\usepackage[margin=0.75in]{geometry}

% Don't number sections
\setcounter{secnumdepth}{0}

% Restrict table of contents to parts through subsubsections
\setcounter{tocdepth}{3}

\hypersetup{
    colorlinks,
    citecolor=black,
    filecolor=black,
    linkcolor=black,
    urlcolor=black
}
\title{SRS \\
Flight Planning
}
\author{ Cedrick Cooke
    \and Ian Littke
    \and Owen Roth-Lerner
    \and Sander Scherman Garzon
    \and Justine O'Neil
}
\date{Version 0.1 \\ 2017-02-21}

\begin{document}
\maketitle

\tableofcontents
\raggedright
\section*{Summary of Changes}
\begin{tabularx}{\textwidth}{|l|r|X|l|}
\hline
Editor & Revision & Description & Date \\ \hline \hline
Ian Littke & 0.1 & Initial document blocking & 2017-02-21 \\ \hline
\end{tabularx}

\section{Introduction}
  \subsection{Purpose}
    This SRS describes the software functional and nonfunctional requirements
    for release 1.0 of the Flight Plan Editor.
    This document is intended to be used by the memers of the project team that will implement
    and verify th correct functioning of the system.
    Unless otherwise noted, all requirements specified here are high priority
    and committed for release 1.0.
  \subsection{Document Conventions}
  This document features some terminology which readers may be unfamiliar with.
  See \hyperref[sec:glossary]{Appendix \ref{sec:glossary} (Glossary)} for a list of these terms and their definitions.
  The format of this SRS is simple. Bold face and indentation is used on general topics and or specific points of interest.
  The remainder of the document will be written using the standard font in \LaTeX.
  \subsection{Intended Audience and Reading Suggestions}
  This document is intended for all individuals participating in the initial release of Flight Plan Editor.
  Including, but not limited to, all project team members; Aran Clauson, team mentor; and Martin Kjelstad, CAP contact point.

  \subsection{Project Scope}
  The Flight Planning Editor will allow flight planners in the Civilian Air Patrol (CAP)
  to plan their flights in an easy fashion and store them in a Secure Digital (SD) card.
  A detailed project description is available in the \hyperref[sec:ref]{\textit{Flight Plan Editor Vision and Scope Document}}.
  The section in that document titled "Scope of Initial and Subsequent Releases" lists the features that are
  scheduled for full or partial implementation in this release.
  \subsection{References}\label{sec:ref}
  \begin{tabularx}{\textwidth}{l|X}
    \hline
    Vision and Scope & \url{https://github.com/cedrickcooke/flight-planning/tree/master/vision-scope}\\
    FAA VFR Raster Charts & \url{https://www.faa.gov/air_traffic/flight_info/aeronav/digital_products/vfr/} \\
    \hline
  \end{tabularx}

\section{Overall Description}
  This section gives an overview of the functionality of the product.
  It describes the informal requirements and is used to establish a context for the technical
  requirements specification in the next section.
\subsection{Product Perspective}
  \subsection{Product Features}

  \subsection{User Classes and Characteristics}
    The user is expected to have knowledge of flying area and can read charts.
    The user shall also be expected to have lattitude-longitude coordinates of
    waypoints of interest to include in the flight plan.
  \subsection{Operating Environment}
    \begin{tabularx}{\textwidth}{|l|X|}
      \hline
      OE-1: & The Flight Plan Editor shall operate with the following Operating Systems:
              Microsoft Windows 7, Microsoft Windows 8 and Microsoft Windows 10\\
      \hline
    \end{tabularx}
  \subsection{Design and Implementation Constraints}
    \begin{tabularx}{\textwidth}{|l|X|}
      \hline
      CO-1: & The system's design and code shall be written in C\# \\ \hline
      CO-2: & All flight plans shall be recorded in XML\\
      \hline
    \end{tabularx}
  \subsection{User Documentation}
    \begin{tabularx}{\textwidth}{|l|X|}
      \hline
      UD-1: & The system shall provide hierarchiacal and cross-linked help system in HTML that
              describes and illustrates all system functions.\\ \hline
      UD-2: & The system shall also offer tooltips to assist in user functions \\
      \hline
    \end{tabularx}
  \subsection{Assumptions and Dependencies}
    \begin{tabularx}{\textwidth}{|l|X|}
      \hline
      AS-1: & Flight operators can map around restricted areas/altitude\\ \hline
      AS-2: & Sectional map(s) are located in the proper location\\ \hline
      DE-1: & Windows operating system\\ \hline
      DE-2: & SD Card supplied is of a certain make/model\\ \hline
      DE-3: & Flight plan is used in Garmin G1000\\ \hline
      \hline
    \end{tabularx}

\section{System Features}
  \subsection{Display Sectional Map}
    \subsubsection{Description and Priority}
      The application will display the sectional map as the main interface.
      The map will respond to input from lattitude/longitude coordinate input as well
      as having the ability to pan around the map.
    \subsubsection{Stimulus/Response Sequences}
      \begin{description}
        \item[Step 1]
      \end{description}
    \subsubsection{Functional Requirements}

  \subsection{Input Destination Coordinates}
    \subsubsection{Description and Priority}
    \subsubsection{Stimulus/Response Sequences}
      \begin{description}
        \item[Step 1]
      \end{description}
    \subsubsection{Functional Requirements}

    \subsection{Distinguish Wapyoint Coordinates}
      \subsubsection{Description and Priority}
      \subsubsection{Stimulus/Response Sequences}
        \begin{description}
          \item[Step 1]
        \end{description}
      \subsubsection{Functional Requirements}

      \subsection{Draw Flight Path on Section Map Between Coordinates}
        \subsubsection{Description and Priority}
        		The user should have the ability to not only add or delete points, but change in which
        		order he connects one point to another. This allows the user to actually route between 
        		points rather than have to manually navigate between them. (Priority = High)
        \subsubsection{Stimulus/Response Sequences}
            Upon selection/editing a flight plan, the program will pull up a list of previous flight
            points and the path between them on a map. These points will be pathed in the order the
            appear on the list, and changing the order will change what point each point connects to.
            The user will be able to see this editing of order (and thereby flight path) automatically
            on the map.
        \subsubsection{Functional Requirements}
			\begin{tabularx}{\textwidth}{|l|X|}
    				\hline
    				FR-1: & The point list shall have ability to edit ordering of lat/long coordinates\\ \hline
     			FR-2: & If adding new lat/long coordinates, the previous coord will connect to new one.\\ \hline
      			FR-3: & If coordinate is last on list, the path will end there.\\ \hline
      			FR-4: & The map shall update to show new path as the list is edited/extended.\\ \hline
    			\end{tabularx}
        \subsection{Save Flight Plan in XML to SD Card}
          \subsubsection{Description and Priority}
          \subsubsection{Stimulus/Response Sequences}
            \begin{description}
              \item[Step 1]
            \end{description}
          \subsubsection{Functional Requirements}

\section{External Interface Requirements}
  \subsection{User Interfaces}
    \begin{tabularx}{\textwidth}{|l|X|}
      \hline
      UI-1: & The UI shall have ability to input lattitude/longitude coordinates\\ \hline
      UI-2: & The UI shall display a chart that can be moved and arranged\\ \hline
      UI-3: & The UI shall have a way to load and save XML flight plan data\\ \hline
    \end{tabularx}
  \subsection{Hardware Interfaces}
    \begin{tabularx}{\textwidth}{|l|X|}
      \hline
      HI-1: & SD Card writer with SD Card\\
      \hline
    \end{tabularx}

  \subsection{Software Interfaces}
    \begin{tabularx}{\textwidth}{|l|X|}
      \hline
      SI-1: & stuff here \\
      \hline
    \end{tabularx}

  \subsection{Communications Interfaces}
  No communications interfaces have been identified.

\subsection{Other Nonfunctional Requirements}
  \subsection{Performance Requirements}
    \begin{tabularx}{\textwidth}{|l|X|}
      \hline
      PE-1: & The UI shall refactor the tiled maps within 1 minutes\\ \hline
    \end{tabularx}

  \subsection{Safety Requirments}
  No safety requirements have been identified.
  \subsection{Security Requirements}
  No security requirements have been identified.
  \subsection{Software Quality Attributes}

\section{Other Requirements}
  \subsection{Key Milestones}
  \begin{tabularx}{\textwidth}{l c l}
    \hline
    \textbf{Milestone} & \textbf{Deadline} & \textbf{Comments}\\
    \hline
    \textit{SRS Document 0.1} & 2/24/2017 & Rough SRS Due \\
    \textit{Vision and Scope Document 1.0} & 4/??/2017 & Vision and Scope finalization \\
    \textit{SRS Document 1.0} & 4/??/2017 & Initial SRS Due \\
    \textit{Release 1.0} & 12/??/2017 & Release Application to CAP \\
    \hline
  \end{tabularx}
\appendix
\section{Glossary} \label{sec:glossary}
\begin{description}[style=nextline, leftmargin=10mm, topsep=0mm,noitemsep]
      \item[CAP] \hfill \ Civilian Air Patrol
      \item[FAA] \hfill \ Federal Aviation Administration
      \item[Garmin G1000] \hfill \ Integratead flight instrumentation system used in planes
      \item[GeoTIFF] \hfill \ Metadata standard which allows georeferenceing information to be embedded within a TIFF file
      \item[GPS] \hfill \ Global Positioning System
      \item[Sectional] \hfill \ Sectioned chart covering a section of an area
      \item[SRS] \hfill \ Software Requirements Specification
      \item[SD Card] \hfill \ Secure Digital Card
      \item[TIFF] \hfill \ Tagged Image File Format
      \item[VFR] \hfill \ Visual Flight Rules
      \item[XML] \hfill \ eXtensible Markup Language
  \end{description}

\section{Analysis Models}
\section{Issues List}



\bibliography{citations}{}
\bibliographystyle{plain}
\end{document}
