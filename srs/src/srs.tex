\documentclass[12pt, letterpaper]{article}
% Reduce the page margins to 3/4 of an inch
\usepackage[margin=0.75in]{geometry}

% Don't number sections
\setcounter{secnumdepth}{0}

% Restrict table of contents to parts through subsubsections
\setcounter{tocdepth}{3}

\usepackage{enumitem}
\title{SRS \\
Flight Planning
}
\author{ Cedrick Cooke
    \and Ian Littke
    \and Owen Roth-Lerner
    \and Sander Scherman Garzon
    \and Justine O'Neil
}
\date{Version 0.1 \\ 2017-02-18}

\begin{document}
\maketitle

\tableofcontents

\section*{Summary of Changes}
\begin{tabularx}{\textwidth}{|l|r|X|l|}
\hline
Editor & Revision & Description & Date \\ \hline \hline
Ian Littke & 0.1 & Initial document blocking & 2017-02-18 \\ \hline
\end{tabularx}

\section{Introduction}
  \subsection{Purpose}
    This SRS describes the software functional and nonfunctional requirements
    for release 1.0 of the Flight Plan Editor.
    This document is intended to be used by the memers of the project team that will implement
    and verify th correct functioning of the system.
    Unless otherwise noted, all requirements specified here are high priority
    and committed for release 1.0.
  \subsection{Document Conventions}
  \subsection{Intended Audience and Reading Suggestions}
  This SRS is 
  
  \subsection{Project Scope}
  The Flight Planning Editor will allow flight planners in the Civilian Air Patrol (CAP)
  to plan their flights in an easy fashion and store them in a Secure Digital (SD) card.
  A detailed project description is available in the \textit{Flight Plan Editor Vision and Scope Document}.
  The section in that document titled "Scope of Initial and Subsequent Releases" lists the features that are
  scheduled for full or partial implementation in this release.
  \subsection{References}

\section{Overall Description}
  This section gives an overview of the functionality of the product.
  It describes the informal requirements and is used to establish a context for the technical
  requirements specification in the next section.
\subsection{Product Perspective}
  \subsection{Product Features}

  \subsection{User Classes and Characteristics}
    The user is expected to have knowledge of flying area and can read charts.
    The user shall also be expected to have lattitude-longitude coordinates of 
    waypoints of interest to include in the flight plan.
  \subsection{Operating Environment}
    \begin{tabularx}{\textwidth}{|l|X|}
      \hline
      OE-1: & The Flight Plan Editor shall operate with the following Operating Systems: 
              Microsoft Windows 7, Microsoft Windows 8 and Microsoft Windows 10\\
      \hline
    \end{tabularx}
  \subsection{Design and Implementation Constraints}
    \begin{tabularx}{\textwidth}{|l|X|}
      \hline
      CO-1: & The system's design and code shall be written in C\#\\ \hline
      CO-2: & All flight plans shall be recorded in XML\\
      \hline
    \end{tabularx}
  \subsection{User Documentation}
    \begin{tabularx}{\textwidth}{|l|X|}
      \hline
      UD-1: & The system shall provide hierarchiacal and cross-linked help system in HTML that 
              describes and illustrates all system functions.\\ \hline
      UD-2: & The system shall also offer tooltips to assist in user functions \\
      \hline
    \end{tabularx}
  \subsection{Assumptions and Dependencies}
    \begin{tabularx}{\textwidth}{|l|X|}
      \hline
      AS-1: & User is using an appropriate SD Card compatible with the Garmin G1000\\ \hline
      AS-2: & User is expected to update charts from the FAA as they become available\\
      \hline
    \end{tabularx}

\section{System Features}
  \subsection{System Feature X}
    \subsubsection{Description and Priority}
    \subsubsection{Stimulus/Response Sequences}
    \subsubsection{Functional Requirements}

\section{External Interface Requirements}
  \subsection{User Interfaces}
    \begin{tabularx}{\textwidth}{|l|X|}
      \hline
      UI-1: & The UI shall have ability to input lattitude/longitude coordinates\\ \hline
      UI-2: & The UI shall display a chart that can be moved and arranged\\ \hline
      UI-3: & The UI shall have a way to load and save XML flight plan data\\ \hline
    \end{tabularx}
  \subsection{Hardware Interfaces}
    \begin{tabularx}{\textwidth}{|l|X|}
      \hline
      HI-1: & SD Card writer with SD Card is jammed\\
      \hline
    \end{tabularx}

  \subsection{Software Interfaces}
    \begin{tabularx}{\textwidth}{|l|X|}
      \hline
      SI-1: & stuff here \\
      \hline
    \end{tabularx}

  \subsection{Communications Interfaces}
  No communications interfaces have been identified.

\subsection{Other Nonfunctional Requirements}
  \subsection{Performance Requirements}
    \begin{tabularx}{\textwidth}{|l|X|}
      \hline
      PE-1: & The UI shall refactor the tiled maps within 1 minutes\\ \hline
    \end{tabularx}

  \subsection{Safety Requirments}
  No safety requirements have been identified.
  \subsection{Security Requirements}
  No security requirements have been identified.
  \subsection{Software Quality Attributes}

\section{Other Requirements}
\appendix
\section{Glossery}
\begin{description}[style=nextline, leftmargin=17mm, topsep=0mm,noitemsep]
      \item[CAP] \hfill \ Civilian Air Patrol
      \item[FAA] \hfill \ Federal Aviation Avionics
      \item[Garmin G1000] \hfill \ Integratead flight instrumentation system used in planes
      \item[GPS] \hfill \ Global Positioning System
  \end{description}

\section{Analysis Models}
\section{Issues List}



\bibliography{citations}{}
\bibliographystyle{plain}
\end{document}
