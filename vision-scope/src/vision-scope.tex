\documentclass[12pt, letterpaper]{article}
% Reduce the page margins to 3/4 of an inch
\usepackage[margin=0.75in]{geometry}

% Import citations for use with BibTeX
\usepackage{cite}

% Don't number sections
\setcounter{secnumdepth}{0}

% Restrict table of contents to parts through subsubsections
\setcounter{tocdepth}{3}


\title{Vision and Scope \\
Flight Planning
}
\author{ Cedrick Cooke
    \and Ian Littke
    \and Owen Roth-Lerner
    \and Sander Scherman Garzon
    \and Justine O'Neil
}
\date{Version 0.1 \\ 2017-01-30}

\begin{document}
\maketitle

\tableofcontents

\section*{Summary of Changes}
\begin{tabularx}{\textwidth}{|l|r|X|l|}
\hline
Editor & Revision & Description & Date \\ \hline \hline
Cedrick Cooke & 0.1 & Initial document blocking & 2017-01-30 \\ \hline
\end{tabularx}

\section{Requirements}
\subsection{Background}
\subsection{Business Opportunity}
\subsection{Business Objectives and Success Criteria}
\subsection{Customer or Market Needs}
Currently, other free flight planning software, (Ex. SkyVector), is web based and therefore is required to be connected to the internet. Our software must be able to be accessable on a desktop, reguardless of internet connection. It must also fit onto a SD card, and be able to be intuitively navicable on a G1000 airplane navigation system. It's longitude and latitude navigation must be accurate for the user-requested locations. 
\subsection{Business Risks}
The primary risk of the software is the threat of innacturate latitude and logitude for a given bridge ifrrastructure, whether it be completely in the wrong area, or switching bridges' locations. The potential for this is probable, as we will be entering numerous locations. However it can be countered by thorough review by peers and other maps to double check locations.

Another risk is convoluted accesibility, causing the use of the program to be impractical or less effeciant than the current method of notebook-to-GPS copying or other software. This is very likely, as this program is not expected to be updated to adress user inconveniances or compete with other future navigation software. This can be mitigated by through user testing and review. 

Finally, our last clear risk is lack of functionality, or complete inability for the program to function on the GPS in flight or be transferred from the desktop to the GPS. This is unlikely, as we will be regularly checking to see that our XML output is identical to the current GPS XML used, and run it before final shipping. 

\section{Vision of Solution}
\subsection{Vision Statement}
\subsection{Major Features}
\subsection{Assumptions and Dependencies}

\section{Scope, Limitations}
\subsection{Scope of Initial Release}
\subsection{Scope of Subsequent Releases}
\subsection{Limitations and Exclusions}

\section{Business Context}
\subsection{Stakeholder Profiles}
\subsection{Project Priorities}
\subsection{Operating Environment}

\bibliography{citations}{}
\bibliographystyle{plain}
\end{document}
