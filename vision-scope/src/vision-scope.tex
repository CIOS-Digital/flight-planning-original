\documentclass[12pt, letterpaper]{article}
% Reduce the page margins to 3/4 of an inch
\usepackage[margin=0.75in]{geometry}

% Import citations for use with BibTeX
\usepackage{cite}

% Don't number sections
\setcounter{secnumdepth}{0}

% Restrict table of contents to parts through subsubsections
\setcounter{tocdepth}{3}


\title{Vision and Scope \\
Flight Planning
}
\author{ Cedrick Cooke
    \and Ian Littke
    \and Owen Roth-Lerner
    \and Sander Scherman Garzon
    \and Justine O'Neil
}
\date{Version 0.1 \\ 2017-01-30}

\begin{document}
\maketitle

\tableofcontents

\section*{Summary of Changes}
\begin{tabularx}{\textwidth}{|l|r|X|l|}
\hline
Editor & Revision & Description & Date \\ \hline \hline
Cedrick Cooke & 0.1 & Initial document blocking & 2017-01-30 \\ \hline
\end{tabularx}

\section{Requirements}
\subsection{Background}
One of the missions which the Civil Air Patrol may have to complete in the event of a serious earthquake is to survey bridge infrastructure to verify its structural integrity.  Pilots are sent on missions to fly over and photograph bridges across the state of Washington.  Currently, CAP pilots are given notebooks containing flight plans with a series of geographic coordinates.  While flying, the pilots manually key the coordinates of their next location into the cockpit navigation system.  The Flight Plan Editor software will allow CAP to generate flight plan datafiles that are compatible with the navigation system so that pilots can select them from the cockpit navigation menu, thereby relieving them of the responsibility of manualy entering location coordinates.

\subsection{Business Opportunity}


\subsection{Business Objectives and Success Criteria}
\subsection{Customer or Market Needs}
\subsection{Business Risks}

\section{Vision of Solution}
\subsection{Vision Statement}
\subsection{Major Features}
\subsection{Assumptions and Dependencies}

\section{Scope, Limitations}
\subsection{Scope of Initial Release}
\subsection{Scope of Subsequent Releases}
\subsection{Limitations and Exclusions}

\section{Business Context}
\subsection{Stakeholder Profiles}
\subsection{Project Priorities}
\subsection{Operating Environment}

\bibliography{citations}{}
\bibliographystyle{plain}
\end{document}
