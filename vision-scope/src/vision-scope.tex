\documentclass[12pt, letterpaper]{article}
% Reduce the page margins to 3/4 of an inch
\usepackage[margin=0.75in]{geometry}

% Import citations for use with BibTeX
\usepackage{cite}

% Don't number sections
\setcounter{secnumdepth}{0}

% Restrict table of contents to parts through subsubsections
\setcounter{tocdepth}{3}


\title{Vision and Scope \\
Flight Planning
}
\author{ Cedrick Cooke
    \and Ian Littke
    \and Owen Roth-Lerner
    \and Sander Scherman Garzon
    \and Justine O'Neil
}
\date{Version 0.2 \\ 2017-02-01}

\begin{document}
\maketitle

\tableofcontents

\section*{Summary of Changes}
\begin{tabularx}{\textwidth}{|l|r|X|l|}
\hline
Editor & Revision & Description & Date \\ \hline \hline
Various & 0.2 & Initial document body and writing & 2017-02-01 \\ \hline
Cedrick Cooke & 0.1 & Initial document blocking & 2017-01-30 \\ \hline
\end{tabularx}

\section{Requirements}
\subsection{Background}
One of the missions which the Civil Air Patrol may have to complete in the event of a serious earthquake is to survey bridge infrastructure to verify its structural integrity.  Pilots are sent on missions to fly over and photograph bridges across the state of Washington.  Currently, CAP pilots are given notebooks containing flight plans with a series of geographic coordinates.  While flying, the pilots manually key the coordinates of their next location into the cockpit navigation system.  The Flight Plan Editor software will allow CAP to generate flight plan datafiles that are compatible with the navigation system so that pilots can select them from the cockpit navigation menu, thereby relieving them of the responsibility of manualy entering location coordinates.

\subsection{Business Opportunity}


\subsection{Business Objectives and Success Criteria}
\subsection{Customer or Market Needs}
\subsection{Business Risks}

\section{Vision of Solution}
\subsection{Vision Statement}
\subsection{Major Features}
\subsection{Assumptions and Dependencies}

\section{Scope, Limitations}
\subsection{Scope of Initial Release}
The initial release will have the bare minimum functionality:
    flight plans for the G1000 will be able to be opened or created,
    and waypoints of the flight may be input or updated by latitude and longitude only.
A prototype interface will exist, but does not have to be representitive of the final interface.

\subsection{Scope of Subsequent Releases}
Subsequent releases will come in three stages to add a graphical interface, locations, and printing of flight plans/maps.
The first subsequent release \emph{must} contain a prototype of the graphical user interface.
The second subsequent release will add printing.
The last release will add pre-defined waypoints for things such as airports.

\subsection{Limitations and Exclusions}
The software must work on Microsoft Windows, and is not required to work on Linux or Mac computers.
The software will assume that the user is correct,
    and the software shall make no attempts to fix errors in the user's input.

\section{Business Context}
\subsection{Stakeholder Profiles}
\subsection{Project Priorities}
\subsection{Operating Environment}

\bibliography{citations}{}
\bibliographystyle{plain}
\end{document}
